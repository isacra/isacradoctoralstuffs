%================================= Resumo e Abstract ========================================
\chapter*{Resumo}


\begin{quotation}

\noindent O processo de caracterização de reservatórios de hidrocarbonetos
consiste na determinação tridimensional e quantitativa da estrutura e das
propriedades petrofísicas das rochas da área de interesse. A utilização de
técnicas de inversão sísmica neste processo tem grande importância, pois
possibilita a obtenção das propriedades elástica e acústica do reservatório.
Inversão sísmica é um problema complexo, não linear, mal determinado e ambíguo,
o que leva à necessidade de alta intervenção de especialistas para restringir o
resultado. Neste processo também é importante a modelagem da incerteza dos
resultados obtidos, fornecendo informações sobre os riscos envolvidos nos
processos de perfuração e na produção de óleo e gás. Na literatura recente,
métodos foram propostos para tentar solucionar o problema da inversão sísmica
com modelagem de incerteza integrada. O uso de ferramental probabilístico
baseado no teorema de Bayes e utilizando restrições \textit{a priori} pode ser
aplicado, em conjunto com o amostrador de Gibbs, para gerar realizações
estocásticas do resultado. Outra proposta destacada na literatura é a inversão
Geoestatística, que também gera realizações à um custo maior do que a anterior,
mas é capaz de modelar melhor a continuidade espacial dos fenômenos. O objetivo
deste projeto é a concepção de um novo modelo integrando o método Bayesiano à
inversão Geoestatística, afim de melhorar a convergência do segundo e
aproveitando sua capacidade de modelar continuidade espacial. O modelo será
também inserido num \textit{framework} de modelagem de incerteza utilizando
redução dimensional, afim de considerar incertezas envolvidas em alguns
parâmetros de entrada fornecidos por especialistas como \textit{wavelets} e
horizontes.

\vspace*{0.5cm}

\noindent Palavras chave: Inversão Sísmica; Modelagem de Incerteza; Inversão Geoestatística; Inversão Bayesiana.

\end{quotation}


\chapter*{Abstract}


\begin{quotation}

\noindent 

The characterization process of hydrocarbon reservoirs entails in determining
the 3D structure and petrophysical properties of the rocks at the area of
interest. The use of seismic inversion techniques in this process have great
importance as it allows obtaining elastic and acoustic properties of the
reservoir. Seismic Inversion is a complex, nonlinear, ill posed and ambiguous
problem, which leads to high expert intervention to restrain the results.
Uncertainty modeling of the results is also important in this process, providing
information over the risks involved in drilling and production of oil and gas.
In recent literature, various methods were proposed to tackle the problem of
seismic inversion with integrated uncertainty modeling. The use of probabilistic
tools based on Bayes theorem and Gibbs sampling can be used to generate
stochastic realizations of the result. Another prominent proposal is the
so-called Geostatistical inversion which also generates stochastic realizations
but at a higher cost, with the advantage of broader spatial modeling
capabilities. This document describes the project to conceive a new model
integrating Bayesian method to Geostatistical inversion, aiming to improve the
convergence of the latter keeping its spatial modeling capabilities. The
proposal will also cover the uncertainty modeling of specialist provided input
parameters using a dimension reduction method called Multi-Dimensional Scaling.



\vspace*{0.5cm}

\noindent Keywords: Seismic Inversion; Uncertainty Modeling; Geostatistical
Inversion; Bayesian Inversion.

\end{quotation}

\null

