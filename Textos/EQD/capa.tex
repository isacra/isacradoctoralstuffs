%================================================================================================
%====================================== FOLHA DE ROSTO ==========================================
%================================================================================================
\thispagestyle{empty}
\begin{center}
\large Universidade Federal de Santa Catarina\\
Departamento de Informática e Estatística\\
Programa de Pós-Graduação em Ciência da Computação
\end{center}

\vspace*{1.5cm}
\begin{center}
\large \aluno
\end{center}


\vspace*{2.3cm}

\begin{center}
{\sc  \tituloTese Modelo de rede neural convolucional para super-resolução da impedância acústica na inversão sísmica}
\end{center}

\vspace*{5.5cm}

\begin{flushright}
\begin{minipage}{9.0cm}
\linespread{1}
Texto entregue como requisito para defesa do Exame de Qualificação de
Doutorado, contendo revisão bibliográfica, problemática, proposta e resultados
prévios.



\vspace*{0.5cm}
Orientador: \nomeProf\\
%Coorientador: \nomeCoorientador

\end{minipage}
\end{flushright}

\null \vfill


\begin{center}
Florianópolis \\ \the\year
\end{center}

%================================================================================================
%============================== Ficha (Somente na versão final) =================================
%================================================================================================
% \newpage
% 
% \begin{center}
% \vspace*{10cm}
% Insira nesta página a sua ficha catalográfica (somente versão final). Obs. É conveniente
% converter o documento fornecido pela BAE (normalmente .doc) em um arquivo .ps. Para a versão
% preliminar da tese (antes da defesa), simplesmente remova essa página.
% \end{center}
% % Observação: a ficha fornecida pela BAE normalmente é fornecida em formato .doc. Existem diversas
% % maneiras de converter .doc em .ps. 
% 
% % Descomente as duas próximas linhas (e comente acima desde o begin{center} até o end{center}) para inserir a ficha catalográfica caso a mesma já  tenha sido convertida para .ps (no caso ficha.ps)
% 
% %\epsfxsize=0.925\columnwidth
% %\epsffile{ficha.ps}
% 
% \null \vfill
%\usepackage{graphics} is needed for \includegraphics

% \begin{figure}[htp]
% \begin{center}
%   \includegraphics[width=\linewidth]{pdf/ficha}
% \end{center}
% \end{figure}

%\includepdf[pages={1},pagecommand={\thispagestyle{plain}}]{pdf/ficha.pdf}
% \newpage

%================================================================================================
%============================== Folha de aprovação (Somente na versão final) ====================
%================================================================================================


% 
% \begin{center}
% \vspace*{10cm}
% Insira nesta página a folha de aprovação fornecida pelo seu programa de pós-graduação (somente versão final). 
% Obs. É conveniente scanear o documento e convertê-lo para o formato .ps. Para a versão
% preliminar da tese (antes da defesa), simplesmente remova essa página.
% \end{center}
% 
% % Escanear a folha de aprovação fornecida pela CPG e converter para .ps ou .eps. A idéia
% % é inserir como uma figura. É muito provável que será necessário fazer ajustes no
% % tamanho, mexendo no comando \epsfxsize
% 
% % Descomente as duas próximas linhas (e comente acima desde o \begin{center} até o \end{center})
% %\epsfxsize=0.985\columnwidth
% %\hspace*{-1.25cm}\epsffile{aprov.eps}
% 
% \null \vfill
%\usepackage{graphics} is needed for \includegraphics

%\includepdf[pages={1},pagecommand={\thispagestyle{plain}}]{pdf/aprova.pdf}

\newpage