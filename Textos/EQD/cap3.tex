\chapter{Revisão da Literatura}
\label{cap:3revisaoliteraria}

Neste capítulo serão apresentadas as revisões sistemáticas relacionadas
ao método de inversão sísmica e dos modelos de super-resolução de imagens.
Esta revisão evidenciou o potencial de pesquisa desta proposta, pois
ficou evidente uma lacuna em métodos de pós-processamento da inversão sísmica.

\section{Métodos de Inversão Sísmica}
É importante ter em mente que, durante a inversão, se trabalha com dois espaços
de representações diferentes: o espaço do modelo e o espaço de dados.
No contexto da inversão sísmica, os dados sísmicos $d$ são representados no espaço
dos dados e a propriedade de impedância acústica $m$ das rochas é representada
no espaço do modelo.

A escolha dos parâmetros do modelo geralmente é não única, de modo que dois conjuntos
de parâmetros diferentes podem ser equivalentes. Independente de uma parametrização
específica, é possível introduzir um espaço de pontos no qual cada ponto representa um modelo
do sistema. Entretanto, para uma abordagem quantitativa, uma parametrização precisa ser definida \cite{tarantola}.
Por outro lado, para obter informações sobre os parâmetros do modelo, é necessário
realizar observações através de experimentos físicos, como por exemplo, a
aquisição sísmica. Este conjunto de dados representa os parâmetros observáveis do sistema,
ou, espaço de dados.

Sob um olhar ingênuo é possível questionar por quê não definir a
função inversa da modelagem direta e calcular, de forma imediata,
os parâmetros do modelo a partir dos espaço dos dados.
No entanto, esses métodos de inversão direta sofrem de instabilidades
devido à ruído e características do problema \citep[p. 50]{sen_livro}. Outra
opção é utilizar tentativa e erro para ajustar os parâmetros até conseguir uma
resposta semelhante aos dados experimentais. Formalmente isto é automatizado
utilizando métodos de otimização. Para tanto, é preciso definir uma função de
custo, ou função objetivo, que mede o ajuste dos dados produzidos pelos
parâmetros do modelo (dado sintético) ao dado medido.
% 
% O objetivo da inversão, no entanto, vai além de encontrar os parâmetros que
% melhor se ajustam aos dados. Quando os dados são ruidosos, o modelo direto não é
% exato e não existem dados suficientes, a inversão tem solução não única, ou
% seja, vários modelos ajustam aos dados de forma equivalente. Consequentemente é
% importante modelar a incerteza envolvida no processo, indicando qual a
% variabilidade dos modelos que se ajustam bem aos dados.

A relação entre o modelo e os dados (modelo direto) é dada por:

\begin{equation}
d = G(m_v) + e
\end{equation}
onde $G(\cdot)$ é uma função não linear, e assume-se que um ruído $e$ está
presente. Em teoria o ruído é uma interferência aleatória que não se tem
controle, na prática se considera ruído tudo que não é explicado pela função
$G$, e.g. imprecisões no modelo físico e problemas com filtragem e processamento
dos dados.

\section{Inversão Sísmica Linear e Não Linear}

% O modelo mais utilizado para aproximar a função $G$, no caso da inversão
% sísmica, é o modelo convolucional. No caso discreto a convolução é dada pelo
% produto acumulado de cada amostra do vetor de refletividades por todas as
% amostras da \textit{wavelet}. Portanto pode ser representado por uma operação
% matricial:
% 
% \begin{equation}
% \label{eq:sismDiscreta}
% \mathbf{d = Gr}
% \end{equation}
% onde $\mathbf{G}$ é uma matriz convolucional construída utilizando uma
% \textit{wavelet} e $\mathbf r$ o vetor das refletividades definido por:
% 
% \begin{equation}
% \label{eq:refletDiscreta}
% r(t)=\frac{z(t+\delta t) - z(t)}{z(t+\delta t) + z(t)}
% \end{equation}
% o que torna não linear a relação entre impedância $\mathbf z$ e o dado sísmico.
% Uma aproximação válida quando valores de refletividades não ultrapassam $0.3$ é:
% 
% 
% \begin{equation}
% r(t) = \frac{1}{2}\Delta \ln(z(t))
% \label{eq:lnz}
% \end{equation}
% 
% 
% Utilizando estas aproximações para o modelo direto, a alternativa mais objetiva
% é incorporar as aproximações na matriz $\mathbf{G}$ e invertê-la para obter
% $\mathbf{\ln(z)}$, por fim aplicar o exponencial para obter $\mathbf{z}$.
% Neste caso temos os seguintes problemas: existência; unicidade; estabilidade; e
% robustez \citep[p. 56-57]{sen_livro}. Utilizando a formulação de mínimos
% quadrados também é possível resolver sistemas sobredeterminados, solucionando
% problemas com a melhor estimativa possível no sentido de minimizar o erro
% quadrático (norma $L_2$). Apesar de ser uma solução mais geral, ainda é
% utilizado somente um critério de ajuste aos dados, o que não possibilita a
% inserção de conhecimento \textit{a priori}. É possível regularizar o método de
% mínimos quadrados, mas ainda não se tem muita liberdade para inserir
% conhecimentos \textit{a priori} e de outras fontes \citep{clappRegLeast3D}.
% 
% 
% Quando não é possível o uso da aproximação da Equação \ref{eq:lnz}, o problema
% deve ser abordado utilizando métodos de otimização não linear. Com isso os erros
% devido às aproximações do modelo \textit{forward} diminuem, mas a otimização se
% torna mais custosa. Como a relação entre os dados e os parâmetros é não linear, a
% função objetivo a ser minimizada irá possuir mínimos locais, tornando necessário
% o uso de métodos de otimização global. Esta prática está bem documentada na
% literatura de inversão, como o uso de \textit{simulated annealing}
% \citep{max_inv_simulated}, de algoritmos genéticos \citep{MallickGeneticInve} e
% enxame de partículas \citep{zhe_nonlinear}. 
% 
% 
% %falar mais das vantagens e desvantagens do nao linear
% 
% Outra forma de inversão presente na literatura é a elástica
% \citep{azevedo2013_avoinv,Buland01012003}. Nesse tipo de inversão os dados
% sísmicos estão em um nível diferente de processamento onde os traços sísmicos
% são empilhados em intervalos de ângulos de incidência. Com isso é possível
% inverter para Vp (velocidade primária/compressional), Vs (velocidade
% secundária/cisalhante) e $\rho$ (densidade), ao invés de somente impedância
% acústica. A Equação \ref{eq:elastica} modela a relação da refletividade $c_{pp}$
% com Vp ($\alpha$), Vs ($\beta$) e $\rho$ para cada ângulo disponível.
% 
% \begin{equation}
% c_{pp}(\theta) = a_\alpha(\theta)\frac{\Delta\alpha}{\bar{\alpha}} +
% a_\beta(\theta) \frac{\Delta\beta}{\bar{\beta}} +
% a_\rho(\theta) \frac{\Delta\rho}{\bar{\rho}}
% \label{eq:elastica}
% \end{equation}
% 
% onde:
% \begin{equation}
% a_\alpha(\theta)= \frac{1}{2}(1+\tan^2\theta),
% \end{equation}
% 
% \begin{equation}
% a_\beta(\theta)= -4\frac{\bar{\beta}^2}{\bar{\alpha}^2}\sin^2\theta,
% \end{equation}
% 
% \begin{equation}
% a_\rho(\theta) = \frac{1}{2} \left (  
% 1-4\frac{\bar{\beta}^2}{\bar{\alpha}^2}\sin^2\theta \right ).
% \end{equation}
% 
% Adicionalmente $\bar{\alpha}$, $\bar{\beta}$ e $\bar{\rho}$ são as respectivas
% médias sobre a interface; $\Delta\alpha$, $\Delta\beta$ e $\Delta\rho$ são os
% contrastes e $\theta$ o ângulo médio de reflexão. Essas propriedades são
% importantes, pois a velocidade secundária é indicador de hidrocarbonetos por não
% se propagar em meio líquido, desta forma áreas com presença destes compostos se
% destacam numa imagem de Vs, podendo indicar interfaces rocha/óleo e água/óleo.
% As metodologias para resolução do problema são semelhantes, mas neste caso
% aumentando a dimensão dos dados e números de parâmetros a serem estimados
% \citep{Buland01012003}.

\subsection{Máximo \textit{a posteriori}}
\label{sec:map}

% A inversão por Máximo \textit{a posteriori} (MAP)
% \citep{Buland01012003,leandroGRSL} é realizada para cada traço individualmente.
% Baseado no modelo convolucional e assumindo que o ruído presente nos dados é
% Gaussiano, o vetor das sísmicas experimentais $\boldsymbol{d}$, é modelado pela
% distribuição de probabilidade:
% 
% \begin{equation}
% p(\boldsymbol{d}|\boldsymbol{\mu_{d}},\boldsymbol{\Sigma_{d}}) =
% N(\boldsymbol{\mu_{d}},\boldsymbol{\Sigma_{d}}),
% \end{equation}
% onde $\boldsymbol{\mu_{d}} = \boldsymbol{Gm}$ é o vetor com a sísmica
% sintética e $\boldsymbol{\Sigma_{d}}$ é a matriz de covariância do ruído da
% sísmica, a qual é definida conforme a confiabilidade que o especialista tem no
% dado sísmico ou seu nível de ruído. Geralmente se utiliza uma matriz diagonal
% com mesma variância para todos os elementos.
% 
% Para o vetor modelo $\boldsymbol{m}$ com o logaritmo natural da impedância
% acústica, considerou-se também uma distribuição normal:
% 
% \begin{equation}
% p(\boldsymbol{m}|\boldsymbol{\mu_{m}},\boldsymbol{\Sigma_{m}}) =
% N(\boldsymbol{\mu_{m}},\boldsymbol{\Sigma_{m}}),
% \end{equation} 
% no qual $\boldsymbol{\mu_{m}}$ é um vetor contendo a baixa frequência do
% logaritmo natural da impedância. Este dado é outra informação adicional que é
% fornecida pelo especialista via análise de velocidades ou interpolando
% dados de poços por Krigagem. Os componentes da matriz de covariância
% $\boldsymbol{\Sigma_{m}}$ foram definidos conforme \citep{leandroGRSL}:
% 
% \begin{equation}
% \label{eq:correlVert}
% \boldsymbol{\nu}_{t,t'} = \sigma_{m}^{2} exp \left (-\frac{(t-t')^2)}{L^2}
% \right ),
% \end{equation} 
% que define a correlação entre as componentes de $\boldsymbol{m}$ no tempo $t$ e
% $t'$, na qual $\sigma_{m}^{2}$ é a variância da impedância acústica calculada
% nos dados de poços sem a baixa frequência, e $L$ é a distância de correlação
% vertical a ser imposta ao resultado.
% 
% Neste arcabouço a média e variância posterior para cada traço podem ser
% calculadas analiticamente via \citep{leandroGRSL}:
% 
% \begin{equation}
% \label{eqn:mapSolution}
% \boldsymbol{\mu}_{m|} = \boldsymbol{\mu}_{m} + \boldsymbol{\Sigma}_{m}\boldsymbol{G}^{T}(\boldsymbol{G\Sigma}_{m}\boldsymbol{G}^{T}+\boldsymbol{\Sigma}_{d})^{-1}\left ( \boldsymbol{d}_{o} - \boldsymbol{G\mu}_{m} \right ),
% \end{equation}
% \begin{equation}
% \boldsymbol{\Sigma}_{m|} = \boldsymbol{\Sigma}_{m} - \boldsymbol{\Sigma}_{m}\boldsymbol{G}^{T}(\boldsymbol{G\Sigma}_{m}\boldsymbol{G}^{T}+\boldsymbol{\Sigma}_{d})^{-1}\boldsymbol{G\Sigma}_{m}.
% \end{equation} 
% onde o cálculo da matriz inversa acima pode ser aproveitado para vários traços
% de uma região de interesse em certos casos, ou seja, quando as matrizes de
% covariância possam ser assumidas iguais para todos os traços da sísmica da região.
% Desta forma alteram-se a sísmica $\mathbf{d}_0$ e a baixa frequência
% $\boldsymbol{\mu_m}$ obtendo-se a média posterior para o traço desejado.
% 
% 
% Utilizando esta metodologia a solução é representada por uma distribuição
% posterior Gaussiana com expressões explícitas para o valor esperado e para a
% covariância. Não são necessárias iterações para ajuste do modelo, tornando o
% método eficiente e útil em casos de uso reais, possibilitando o especialista
% alterar parâmetros e avaliar o resultado em tempo real numa pequena área.
% Satisfeito com a parametrização, o método é aplicado a todo o campo. A matriz de
% covariância posterior indica a incerteza presente no resultado, não é necessário
% definir a tolerância de ajuste aos dados explicitamente, mas é preciso definir a
% matriz de covariância \textit{a priori} do resultado esperado, ou seja, é
% preciso ter conhecimento, mesmo que de forma grosseira, das correlações espaciais e
% variâncias que se espera do resultado. Ao final, a covariância posterior é
% calculada.
% 
% Uma desvantagem destacada na literatura é a dificuldade em inserir modelos de
% continuidade mais abrangentes, pois é necessário incluir as covariâncias entre
% traços vizinhos, aumentando as matrizes de covariância e a matriz a ser
% invertida. Inserindo as correlações horizontais também inviabiliza aproveitar o
% cálculo da matriz inversa, pois desta forma o resultado da inversão precisa ser
% calculado para todos os pontos a serem invertidos ao mesmo tempo.
% Atualmente é inserida somente covariância entre amostras no mesmo traço, ou
% seja, somente na direção vertical.
% 
\section{Métodos Inteligentes de Super-resolução de Imagens}

A super-resolução tem aplicações nas mais diferentes áreas...

A super-resolução tem um papel importante na visão computacional \cite{DongLoy2016}.
Métodos baseados em interpolação são fáceis de implementar e amplamente utilizados,
entretanto estes métodos sofrem de falta de expressividade, uma vez que modelos lineares
não são capazes de expressar dependências complexas entre as entradas e as saídas \cite{HsiehAndrews1978}.
Na prática tais métodos falham na tentativa de prever adequadamente detalhes de alta frequência
levando a saídas de alta resolução borradas. Efeito semelhante ocorre durante a inversão sísmica,
na qual as imagens resultantes apresentam resolução limitada e contornos borrados. Assim, um modelo
de rede convolucional foi testado para atribuir às imagens de inversão sísmica para impedância
elementos de alta frequência.

A proposta para pesquisar um modelo baseado em redes convolucionais para pós-processamento
da inversão sísmica se fundamenta na lacuna existente de métodos...
\textbf{A revisão da literatura foi realizada com as seguintes palavras-chaves:}
A revisão foi realizada sistematicamente nos seguintes periódicos:

Foram selecionados os documentos que datam de, pelo menos 10 anos. Ficou evidente que o problema
de super-resolução vem sendo tratado por diferentes abordagens....
Mais recentemente, os avanços das pesquisas do Google em \textit{Deep Learning} disponibilizaram
ferramentas de implementação de diferentes algoritmos de aprendizagem de máquina. Dentre estas
ferramentas está o \textit{Framework} de \textit{Deep Learning} TensorFlow, no qual os modelos
de redes convolucionais podem ser implementadas e testadas.

\textbf{Dentre os textos encontrados é possível destacar:}

\textbf{porém não foram encontrados trabalhos que abordem o problema de aumento de resolução de imagens de propriedades petrofísicas
pós-inversão.}

\section{Resumo}

Neste capítulo foi revisado o estado da arte em inversão sísmica acústica
com modelagem de incerteza. Pontos críticos dos métodos foram considerados e
identificados para pesquisa futura. O próximo capítulo irá definir a proposta de
pesquisa, apresentar o plano de trabalho e concluir com as perspectivas de
contribuição.
