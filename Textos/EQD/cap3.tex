\chapter{Revisão da Literatura}
\label{cap:3revisaoliteraria}

Neste capítulo serão apresentadas as revisões sistemáticas relacionadas
ao método de inversão acústica e dos modelos de super-resolução de imagens.
Esta revisão evidenciou o potencial de pesquisa desta proposta, pois
apresenta uma lacuna em métodos de pós-processamento da inversão sísmica.

\section{Métodos de Inversão Sísmica}
É importante ter em mente que, durante a inversão, as operações são realizadas
sobre dois espaços de representações diferentes: o espaço do modelo e o espaço de dados.
No contexto da inversão sísmica, os dados sísmicos $d$ são representados no espaço
dos dados e a propriedade de impedância acústica das rochas é representada
no espaço do modelo $m$.
A escolha dos parâmetros do modelo geralmente é não única, de modo que dois conjuntos
de parâmetros diferentes podem ser equivalentes.
Entretanto, para uma abordagem quantitativa, uma parametrização precisa ser definida \cite{tarantola} e,
no contexto da inversão acústica, o parâmetro adotado é a impedância acústica.
Para obter informações sobre os parâmetros do modelo, é necessário
realizar observações através de experimentos físicos, como por exemplo, a
aquisição sísmica. Este conjunto de dados representa os parâmetros observáveis do sistema,
ou espaço de dados, e representa o ponto de partida para a inversão.

Sob um olhar ingênuo é possível questionar por quê não definir a
função inversa da modelagem direta e calcular, de forma imediata,
os parâmetros do modelo a partir dos espaço dos dados.
No entanto, os métodos de inversão direta sofrem de instabilidades
devido ao ruído e características do problema \citep[p. 50]{sen_livro}. Outra
opção é utilizar tentativa e erro para ajustar os parâmetros até conseguir uma
resposta semelhante aos dados experimentais. Formalmente isto é automatizado
utilizando métodos de otimização. Para tanto, é preciso definir uma função de
custo, ou função objetivo, que mede o ajuste dos dados produzidos pelos
parâmetros do modelo (dado sintético) ao dado medido.

\subsection{Inversão Sísmica Linear e Não Linear}
O conteúdo desta seção apresenta as suposições de linearidade necessárias
que tornam a inversão acústica um processo analítico e computacionalmente
eficiente. O detalhamento matemático pode ser consultado em \cite{Figueiredo2014}.

Para entender o processo de inversão sísmica, é conveniente ter em mente
que os problemas inversos podem ser classificados de acordo com a natureza
do relacionamento entre os dados e o modelo, e de acordo com o comportamento da função objetivo.
Assim, eles podem ser: linear, fracamente não-linear, quasi-linear e não-linear.
Na maioria dos problemas geofísicos o operador direto $G$ é não-linear.
Como nos algoritmos de aprendizagem de máquina, na inversão sísmica
a não-linearidade implica em uma função de custo com forma complicada,
possivelmente com mínimos locais.
Por outro lado, se o operador $G$ for aproximadamente linear, a
função de erro se tornará convenientemente quadrática em relação a perturbações
no espaço do modelo. A maior parte da teoria de inversão é baseada em problemas
de inversão linear e, em muitas aplicações, ela é 
adequada para representar a natureza do sistema \cite{sen_livro}.

O modelo sísmico direto pode ser representado pelo modelo convolucional dado por:
\begin{equation}
d(t) = \int_{-\infty}^{\infty} s(\tau) r(t - \tau)\mathrm{d}\tau + e_{d}(t)
\label{eq:conmodel}
\end{equation}
onde $d(t)$ é o traço sísmico, $s(t)$ é a \textit{wavelet}, $e(t)$ é
um ruído aleatório e $r(t)$ é o refletividade.
A representação discreta para o modelo convolucional do dado sísmico é
dado pela operação matricial: 
\begin{equation}
\label{eq:sismDiscreta}
\mathbf{d = Sr + e}
\end{equation}
onde $\mathbf{S}$ é uma matriz convolucional construída utilizando uma
\textit{wavelet} e $\mathbf{r}$  é a matriz de refletividades.

Como já mencionado, a relação entre o pulso sísmico e a
propriedade de impedância acústica é não-linear.
Para escapar da problemática da não-linearidade do operador direto, é
necessário aproximar linearmente o pulso sísmico da impedância acústica.
Para isto, duas medidas são necessárias: a primeira é admitir a 
refletividade como o logaritmo da impedância acústica (equação \ref{eq:lnz}).

\begin{equation}
r(t) = \frac{1}{2}\Delta \ln(z(t))
\label{eq:lnz}
\end{equation}

Esta aproximação é válida para valores de refletividade menores que $0.3$.
A segunda medida, é adotar um operador diferencial $\textbf{D}$. Assim,
se define o operador linear $\textbf{G=(1/2)SD}$ e o modelo $m=ln(z)$.
Com isto, a relação entre o dado sísmico e o parâmetro do modelo (impedância acústica)
se torna linear por:

\begin{equation}
\label{eq:sismDiscreta2}
\mathbf{d = Gm + e}
\end{equation}
Em teoria, o ruído é uma interferência aleatória que não se tem
controle, na prática se considera ruído tudo que não é explicado pela função
$G$, e.g. imprecisões no modelo físico e problemas com filtragem e processamento
dos dados. Com isto, o problema da inversão acústica se torna não-linear e passa
a ter representação matricial.

% Quando não é possível o uso da aproximação da Equação \ref{eq:lnz}, o problema
% deve ser abordado utilizando métodos de otimização não linear. Com isso os erros
% devido às aproximações do modelo \textit{forward} diminuem, mas a otimização se
% torna mais custosa. Como a relação entre os dados e os parâmetros é não linear, a
% função objetivo a ser minimizada irá possuir mínimos locais, tornando necessário
% o uso de métodos de otimização global. Esta prática está bem documentada na
% literatura de inversão, como o uso de \textit{simulated annealing}
% \citep{max_inv_simulated}, de algoritmos genéticos \citep{MallickGeneticInve} e
% enxame de partículas \citep{zhe_nonlinear}. 
% 
% 
% %falar mais das vantagens e desvantagens do nao linear
% 
% Outra forma de inversão presente na literatura é a elástica
% \citep{azevedo2013_avoinv,Buland01012003}. Nesse tipo de inversão os dados
% sísmicos estão em um nível diferente de processamento onde os traços sísmicos
% são empilhados em intervalos de ângulos de incidência. Com isso é possível
% inverter para Vp (velocidade primária/compressional), Vs (velocidade
% secundária/cisalhante) e $\rho$ (densidade), ao invés de somente impedância
% acústica. A Equação \ref{eq:elastica} modela a relação da refletividade $c_{pp}$
% com Vp ($\alpha$), Vs ($\beta$) e $\rho$ para cada ângulo disponível.
% 
% \begin{equation}
% c_{pp}(\theta) = a_\alpha(\theta)\frac{\Delta\alpha}{\bar{\alpha}} +
% a_\beta(\theta) \frac{\Delta\beta}{\bar{\beta}} +
% a_\rho(\theta) \frac{\Delta\rho}{\bar{\rho}}
% \label{eq:elastica}
% \end{equation}
% 
% onde:
% \begin{equation}
% a_\alpha(\theta)= \frac{1}{2}(1+\tan^2\theta),
% \end{equation}
% 
% \begin{equation}
% a_\beta(\theta)= -4\frac{\bar{\beta}^2}{\bar{\alpha}^2}\sin^2\theta,
% \end{equation}
% 
% \begin{equation}
% a_\rho(\theta) = \frac{1}{2} \left (  
% 1-4\frac{\bar{\beta}^2}{\bar{\alpha}^2}\sin^2\theta \right ).
% \end{equation}
% 
% Adicionalmente $\bar{\alpha}$, $\bar{\beta}$ e $\bar{\rho}$ são as respectivas
% médias sobre a interface; $\Delta\alpha$, $\Delta\beta$ e $\Delta\rho$ são os
% contrastes e $\theta$ o ângulo médio de reflexão. Essas propriedades são
% importantes, pois a velocidade secundária é indicador de hidrocarbonetos por não
% se propagar em meio líquido, desta forma áreas com presença destes compostos se
% destacam numa imagem de Vs, podendo indicar interfaces rocha/óleo e água/óleo.
% As metodologias para resolução do problema são semelhantes, mas neste caso
% aumentando a dimensão dos dados e números de parâmetros a serem estimados
% \citep{Buland01012003}.

\subsection{Máximo \textit{a posteriori}}
\label{sec:map}

A teoria mais simples e genérica possível é obtida quando se usa uma
abordagem probabilística \citep{tarantola}. Na solução para a inversão
sísmica, os parâmetros do modelo convolucional \ref{eq:sismDiscreta2}
podem ser representados em termos de suas distribuições de probabilidade.
No modelo estocástico proposto por \cite{Figueiredo2014}, as distribuições
são consideradas normais e multivariadas e são denotadas por $N(\boldsymbol{\mu},\boldsymbol{\Sigma})$.
Assim, assumindo que o ruído $\boldsymbol{e}$ respeita uma distribuição gaussiana,
as distribuições de probabilidade para o vetor dos dados sísmicos experimentais
$\boldsymbol{d}$, para a \textit{wavelet} $\boldsymbol{w}$ e para
o vetor dos parâmetros do modelo $\boldsymbol{m}$ são
definidos, respectivamente, pelas distribuições \ref{eq:psismico}, \ref{eq:pwavelet} e \ref{eq:pmodelo}.

\begin{equation}
\label{eq:psismico}
p(\boldsymbol{d}|\boldsymbol{\mu_{d}},\boldsymbol{\Sigma_{d}}) =
N(\boldsymbol{\mu_{d}},\boldsymbol{\Sigma_{d}})
\end{equation}
Onde $\boldsymbol{\mu_{d}} = \boldsymbol{Gm}$ é o vetor com a sísmica
sintética e $\boldsymbol{\Sigma_{d}}$ é a matriz de covariância do ruído da
sísmica, a qual é definida conforme a confiabilidade que o especialista tem no
dado sísmico ou seu nível de ruído.

\begin{equation}
\label{eq:pwavelet}
p(\boldsymbol{s}|\boldsymbol{\mu_{s}},\boldsymbol{\Sigma_{s}}) =
N(\boldsymbol{\mu_{s}},\boldsymbol{\Sigma_{s}}),
\end{equation} 
Onde o valor esperado da \textit{wavelet} $\boldsymbol{\mu_{s}}$ é definido
como um vetor nulo. Para que o método possa ser aplicado para a inversão
acústica, é necessário estimar uma \textit{wavelet} que possa ser aplicada
no modelo convolucional. Esta estimativa é realizada aplicando este mesmo processo
de inversão na região de ocorrência de um poço, onde a refletividade pode
ser calculada diretamente \citep{Figueiredo2014}. O algoritmo de
Gibbs então é utilizado para amostrar na distribuição posterior da \textit{wavelet} e a 
o valor médio e a incerteza são calculados.

\begin{equation}
\label{eq:pmodelo}
p(\boldsymbol{m}|\boldsymbol{\mu_{m}},\boldsymbol{\Sigma_{m}}) =
N(\boldsymbol{\mu_{m}},\boldsymbol{\Sigma_{m}}),
\end{equation} 
Neste ponto é possível inserir no método de inversão informações \textit{a priori}
que eventualmente estejam disponíveis. Por exemplo, $\boldsymbol{\mu_{m}}$ pode ser
uma matriz de baixas frequências gerada a partir da interpolação da impedância acústica observada em dois poços
já perfurados \citep{Figueiredo2014}.

%Este dado é outra informação adicional que é
% fornecida pelo especialista via análise de velocidades ou interpolando
% dados de poços por Krigagem. Os componentes da matriz de covariância
% $\boldsymbol{\Sigma_{m}}$ foram definidos conforme \citep{leandroGRSL}:



% \begin{equation}
% \label{eq:correlVert}
% \boldsymbol{\nu}_{t,t'} = \sigma_{m}^{2} exp \left (-\frac{(t-t')^2)}{L^2}
% \right ),
% \end{equation} 
% que define a correlação entre as componentes de $\boldsymbol{m}$ no tempo $t$ e
% $t'$, na qual $\sigma_{m}^{2}$ é a variância da impedância acústica calculada
% nos dados de poços sem a baixa frequência, e $L$ é a distância de correlação
% vertical a ser imposta ao resultado.
% 

A inversão por Máximo \textit{a posteriori} (MAP)
\citep{Buland01012003,leandroGRSL} é realizada para cada traço individualmente.
As distribuições condicionais e o modelo convolucional apresentados anteriormente
são as estruturas necessárias para realizar a inversão acústica. O ponto de partida
é a aplicação do próprio método para estimar a \textit{wavelet},
com ela é possível estimar as distribuições de probabilidades envolvidas no modelo.
Em seguida, basta calcular a exponencial do modelo convolucional para obter a distribuição posterior
para o parâmetro do modelo, que no caso em questão é a impedância acústica. Esta distribuição é dada por:

\begin{equation}
p(\boldsymbol{m}|\boldsymbol{d_{o}},\boldsymbol{s},\boldsymbol{\mu_{m}},\sigma_{d}^{2},\sigma_{m}^{2}) = 
N(\boldsymbol{\mu_{m|}},\boldsymbol{\Sigma_{m|}}),
\end{equation} 

Neste arcabouço a média e variância posterior para cada traço podem ser
calculadas analiticamente via \citep{leandroGRSL}:

\begin{equation}
\label{eqn:mapSolution}
\boldsymbol{\mu}_{m|} = \boldsymbol{\mu}_{m} + \boldsymbol{\Sigma}_{m}\boldsymbol{G}^{T}(\boldsymbol{G\Sigma}_{m}\boldsymbol{G}^{T}+\boldsymbol{\Sigma}_{d})^{-1}\left ( \boldsymbol{d}_{o} - \boldsymbol{G\mu}_{m} \right ),
\end{equation}
\begin{equation}
\boldsymbol{\Sigma}_{m|} = \boldsymbol{\Sigma}_{m} - \boldsymbol{\Sigma}_{m}\boldsymbol{G}^{T}(\boldsymbol{G\Sigma}_{m}\boldsymbol{G}^{T}+\boldsymbol{\Sigma}_{d})^{-1}\boldsymbol{G\Sigma}_{m}.
\end{equation} 
onde o cálculo da matriz inversa acima pode ser aproveitado para vários traços
de uma região de interesse em certos casos, ou seja, quando as matrizes de
covariância possam ser assumidas iguais para todos os traços da sísmica da região.
Desta forma alteram-se a sísmica $\mathbf{d}_0$ e a baixa frequência
$\boldsymbol{\mu_m}$ obtendo-se a média posterior para o traço desejado.
  
% 
% Utilizando esta metodologia a solução é representada por uma distribuição
% posterior Gaussiana com expressões explícitas para o valor esperado e para a
% covariância. Não são necessárias iterações para ajuste do modelo, tornando o
% método eficiente e útil em casos de uso reais, possibilitando o especialista
% alterar parâmetros e avaliar o resultado em tempo real numa pequena área.
% Satisfeito com a parametrização, o método é aplicado a todo o campo. A matriz de
% covariância posterior indica a incerteza presente no resultado, não é necessário
% definir a tolerância de ajuste aos dados explicitamente, mas é preciso definir a
% matriz de covariância \textit{a priori} do resultado esperado, ou seja, é
% preciso ter conhecimento, mesmo que de forma grosseira, das correlações espaciais e
% variâncias que se espera do resultado. Ao final, a covariância posterior é
% calculada.
% 

%Finalizar falando do aumento de resolução e puxar o gancho para próxima seção

% Uma desvantagem destacada na literatura é a dificuldade em inserir modelos de
% continuidade mais abrangentes, pois é necessário incluir as covariâncias entre
% traços vizinhos, aumentando as matrizes de covariância e a matriz a ser
% invertida. Inserindo as correlações horizontais também inviabiliza aproveitar o
% cálculo da matriz inversa, pois desta forma o resultado da inversão precisa ser
% calculado para todos os pontos a serem invertidos ao mesmo tempo.
% Atualmente é inserida somente covariância entre amostras no mesmo traço, ou
% seja, somente na direção vertical.
% 

Assim, a inversão por MAP disponibiliza um cubo de impedâncias acústicas
médias, além disto, o nível de resolução das imagens é limitada a resolução
da sísmica e o método depende de um modelo interpolado de baixa resolução
\texit{a priori}. Com isto, as imagens de impedância acústica obtidas
se caracterizam por serem imagens suavizadas, principalmente na região de
transição entre camadas.

\section{Métodos Inteligentes de Super-resolução de Imagens}

A super-resolução tem aplicações nas mais diferentes áreas...

A super-resolução tem um papel importante na visão computacional \citep{DongLoy2016}.
Métodos baseados em interpolação são fáceis de implementar e amplamente utilizados,
entretanto estes métodos sofrem de falta de expressividade, uma vez que modelos lineares
não são capazes de expressar dependências complexas entre as entradas e as saídas \citep{HsiehAndrews1978}.
Na prática tais métodos falham na tentativa de prever adequadamente detalhes de alta frequência
levando a saídas de alta resolução borradas. Efeito semelhante ocorre durante a inversão sísmica,
na qual as imagens resultantes apresentam resolução limitada e contornos borrados. Assim, um modelo
de rede convolucional foi testado para atribuir às imagens de inversão sísmica para impedância
elementos de alta frequência.

A proposta para pesquisar um modelo baseado em redes convolucionais para pós-processamento
da inversão sísmica se fundamenta na lacuna existente de métodos...
A revisão da literatura foi realizada com as seguintes palavras-chaves:
A revisão foi realizada sistematicamente nos seguintes periódicos:

Foram selecionados os documentos que datam de, pelo menos 10 anos. Ficou evidente que o problema
de super-resolução vem sendo tratado por diferentes abordagens....
Mais recentemente, os avanços das pesquisas do Google em \textit{Deep Learning} disponibilizaram
ferramentas de implementação de diferentes algoritmos de aprendizagem de máquina. Dentre estas
ferramentas está o \textit{Framework} de \textit{Deep Learning} TensorFlow, no qual os modelos
de redes convolucionais podem ser implementadas e testadas.

Dentre os textos encontrados é possível destacar:

porém não foram encontrados trabalhos que abordem o problema de aumento de resolução de imagens de propriedades petrofísicas
pós-inversão.

\section{Resumo}

Neste capítulo foi revisado o estado da arte em inversão sísmica acústica
com modelagem de incerteza. Pontos críticos dos métodos foram considerados e
identificados para pesquisa futura. O próximo capítulo irá definir a proposta de
pesquisa, apresentar o plano de trabalho e concluir com as perspectivas de
contribuição.
