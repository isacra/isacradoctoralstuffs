\chapter{Introdução}
\label{cap:1intro}
\pagenumbering{arabic}
% 
Um aspecto importante nas ciências físicas é poder inferir sobre parâmetros
físicos a partir de dados. Em geral, as leis da física disponibilizam os
artefatos necessários para calcular valores de dados, a partir de um modelo.
Este procedimento é conhecido como problema direto (\textit{forward problem}).
A modelagem direta, portanto, inicia com um modelo, sobre o qual um experimento ou processo
é simulado matematicamente. Se o modelo estiver correto, a resposta
obtida deve parecer com dados reais. O processo de inversão faz exatamente o contrário,
consiste em utilizar as medidas efetuadas para inferir os valores de parâmetros que
caracterizam o sistema \citep{tarantola}.

Para entender o problema inverso é conveniente explicar o problema direto antes.
Considere o seguinte exemplo: suponha que uma pedra é lançada em um poço de água.
Após determinado tempo um som é ouvido. É esperado que haja uma relação entre a
profundidade do poço e o tempo entre soltar a pedra e ouvir o som do impacto. Da física,
ocorre a existência de uma relação causa-efeito para este evento, dada por:
\begin{equation}
\label{eq:forw}
T = \sqrt{\frac{1}{5}H} + \frac{1}{340}H
\end{equation}
onde a profundidade $H$ é a causa e o tempo $T$ é o efeito.
Neste caso, o problema direto é calcular o tempo $T$ para ouvir o som, dada profundidade $H$.
A solução pode ser determinada inserindo um valor pra $H$ na equação \ref{eq:forw} e calcular o
valor de $T$. O problema inverso é uma abordagem mais difícil, pois se deseja
saber a profundidade $H$, dado apenas o tempo $T$.

% Para entender o problema inverso é conveniente explicar o problema direto antes.
% Considere o seguinte exemplo: suponha que uma pedra é lançada em um poço de água.
% Após determinado tempo um som é ouvido. É esperado que haja uma relação entre a
% profundidade do poço e o tempo entre soltar a pedra e ouvir o som do impacto. Considere
% a existência de uma relação causa-efeito para este evento, dada por:
% \begin{equation}
% \label{eq:forw}
% T = f(H)
% \end{equation}
% onde a profundidade $H$ é a causa e o tempo $T$ é o efeito.
% Neste caso, o problema direto é calcular o tempo $T$ para ouvir o som, dada profundidade $H$.
% Se a forma da função $f$ é conhecida, a solução pode ser determinada inserindo um valor pra $H$ na equação \ref{eq:forw} e calcular o
% valor de $T$. O problema inverso é uma abordagem mais difícil, pois se deseja
% saber a profundidade $H$, dado apenas o tempo $T$.

No exemplo citado, ambos os problemas, direto e inverso, possuem solução. Entretanto,
a maioria dos problemas inversos recai sobre duas características
comuns que tornam sua solução não-trivial. Primeiro, a não unicidade de solução
(problema não-determinístico), na qual o mesmo conjunto de medidas
observáveis pode resultar de mais de uma configuração de parâmetros. No exemplo
citado, seria como obter a mesma altura $H$ para diferentes tempos
de queda $T$ da pedra. Segundo, a natureza mal-posta do problema inverso, isto é,
uma pequena mudança arbitrária nos valores observados pode causar uma mudança grande
na solução fonte equivalente. Em um paralelo com o exemplo do poço, é como
obter uma grande variação na profundidade, dado uma pequena variação no valor do tempo de queda.

Por conta da sua característica mal-posta, o problema inverso
admite muitas soluções, de modo que representaria um equívoco considerar apenas
uma solução como a mais correta. Via de regra, ao final do processo de inversão é comum
realizar um processo de amostragem dentro do conjunto das possíveis soluções a fim
de obter um estudo sobre elas. Este estudo pode ser uma análise de 
incerteza em torno da média de um conjunto de soluções do problema inverso.

O problema inverso possui um papel de extrema importância em diferentes áreas do conhecimento
como Matemática, Medicina, Física e Geoestatística. Geoestatística é a aplicação de métodos
estatísticos nas ciências da terra. Esta é a disciplina que trata, dentre outros assuntos, da modelagem
e caracterização de reservatórios, cujo tema é de amplo interesse para a indústria de óleo
e gás. Por caracterização de reservatório se entende o processo para obter um modelo de propriedades
petrofísicas (tais como, tipos de contato entre rochas, porosidade e permeabilidade),
em 3-D e alta resolução, e que seja consistente com os dados de que se dispõe \citep{deutsch2002}.
O processo de caracterização de reservatórios possui diferentes
etapas que podem ser descritas em alto nível como na abordagem sequencial a seguir:
\begin{enumerate}
 \item A primeira etapa envolve definição da geometria e estratigrafia dos intervalos do reservatório
 a ser modelado. Ainda, o desenvolvimento de um modelo conceitual de continuidade para
 propriedades como fácies, porosidade e permeabilidade \label{itm:1}.
 \item Modelagem dos tipos de contato entre rochas, mais conhecidos como \textit{facies}\label{itm:2}.
 \item Modelagem da propriedade porosidade com base nos tipos de facies. A modelagem de porosidade
 costuma ser realizada antes da permeabilidade, devido à disponibilidade de dados sísmicos e dados
 amostrais localizados, também chamados de poços\label{itm:3}.
 \item Os modelos 3-D para permeabilidade são atrelados à porosidade e facies anteriormente estabelecidos\label{itm:4}.
 \item Múltiplas realizações, igualmente prováveis, são realizadas por repetição de todo o processo. Embora todas as
 realizações sejam equiprováveis, há realizações mais similares a outras, de modo que a classe à qual pertencem possui
 maior probabilidade\label{itm:5}.
 \item Os modelos são usados como entrada em um simulador ou visualizados e usados como suporte na tomada de decisão\label{itm:6}.
\end{enumerate}

Embora a abordagem anterior utilize alguns termos que, à primeira vista, pareçam incompreensíveis,
sua apresentação contextualiza o ponto do processo de modelagem de
reservatório em que a inversão sísmica acontece (etapa \ref{itm:3}).
Porque os dados medidos são obtidos por sísmica de reflexão,
este método é chamado de inversão sísmica. A inversão sísmica na
modelagem de reservatórios disponibiliza artefatos que são modelos de propriedades de rocha
(propriedades petrofísicas) a partir principalmente, mas não exclusivamente,
da sísmica disponível e de modelos construídos com dados amostrais. Tais artefatos
são visualizados na forma de imagens, de modo que é possível supor que quanto maior
o nível de resolução destas imagens, mais contundente será a justificativa para a tomada de decisão
(etapa \ref{itm:6}).

Atualmente, em relação aos métodos de Aprendizagem de Máquina, \textit{Deep Learning} é o tema em
maior evidência. \textit{Deep Learning} é toda solução que permite aos
computadores aprender a partir da experiência e entender o mundo em
termos de hierarquia de conceitos.
Assim, nos algoritmos de \textit{Deep Learning}, o aprendizado
por experiência (supervisionado) evita a interferência
humana no sentido de especificar formalmente o conhecimento que o computador necessita.
Adicionalmente, a hierarquia de conceitos permite aos computadores aprenderem
conceitos complicados a partir de conceitos mais simples \citep{Gdfl16}.

No campo de
processamento de imagens o método de \textit{Deep Learning} de
maior destaque nos dias atuais é conhecido
como Redes Neurais Convolucionais (RNC). Seu surgimento
data da década de 1980, com aplicação essencialmente no reconhecimento de imagens.
Entretanto, com o advento das Unidades Gráficas de Processamento (GPU) e a maior
disponibilidade de dados para treinamento, as redes convolucionais
são empregadas com sucesso em serviços de busca de imagens, carros auto-dirigíveis,
sistemas de classificação de imagens em video, entre outras aplicações complexas \citep[p. 50]{Buduma15}.

%%Falta referência neste parágrafo
O processo de super-resolução é um método de processamento de imagens e visão computacional.
Consiste em recuperar uma imagem de alta resolução a partir de uma imagem
de baixa resolução. Semelhante ao problema da inversão sísmica, o problema da super-resolução
também é mal-posto, uma vez que pode haver múltiplas soluções em alta resolução para uma dada imagem em baixa
resolução. Em ambos os problemas, uma forma de lidar com esta questão é restringir o espaço
de soluções com informações \textit{a priori} \citep{DongLoy14}. 
Considere $Y$ uma imagem de baixa resolução, por exemplo, uma
imagem interpolada. O objetivo da super-resolução é
recuperar, a partir de $Y$, uma imagem $F(Y)$ que é
o mais similar possível a uma imagem de alta resolução $X$
considerada como imagem que representa a verdade.
De um ponto de vista conceitual, o mapeamento $F$ consiste essencialmente em
realizar três operações:
\begin{enumerate}
 \item Para a imagem de baixa resolução $Y$, extração de mapas de características. Um mapa de características
 pode ser imaginado como um conjunto de sub-imagens com determinadas características da imagem original.
 \item Mapeamento não-linear do mapa de características. Após a operação não-linear, cada característica mapeada
 passa a ter uma representação em alta resolução.
 \item Reconstrução da imagem a partir das representações de alta resolução citadas. Esta nova imagem
 deve ser similar à imagem que representa a verdade $X$.
\end{enumerate}

Um modelo de rede neural convolucional pode pode ser treinada para realizar as etapas listadas anteriormente,
de forma iterativa e para um conjunto de diferentes imagens.
A proposta aqui discutida, sugere incorporar a abordagem das redes convolucionais ao final do processo de inversão.
As imagens obtidas da inversão sísmica podem ser pós-processadas pelo modelo convolucional
a fim de gerar imagens com maior riqueza de detalhes. Como mencionado, o ganho de resolução
nas imagens de propriedades petrofísicas pós-inversão pode conferir maior confiabilidade
na interpretação da solução inversa e, consequentemente na tomada de decisão.

A seção seguinte apresenta a hipótese de pesquisa deste trabalho. Nos capítulos posteriores
serão apresentados os aspectos físicos, de implementação desta proposta e os resultados preliminares.

\section{Hipótese}
A super-resolução das imagens de inversão sísmica pode ser realizada com o aumento da alta frequência na aquisição e
processamento do dado sísmico utilizado na inversão, entretanto, esta é uma tarefa difícil
devido a fatores como atenuação da terra e ruído \citep{Xiaoyu2012}.
Baseado neste fato, a hipótese de pesquisa deste trabalho é de que é possível obter ganho de resolução
na inversão sísmica através da aplicação de um modelo de rede neural convolucional como método de pós-processamento.

\section{Objetivos}

Este trabalho investiga a problemática da super-resolução dos artefatos da inversão sísmica
por meio de um modelo de redes neurais convolucionais.
A abordagem se dá através da incorporação deste modelo de rede neural na última
etapa do processo de inversão sísmica, para alcançar um maior nível de
resolução das imagens de propriedades petrofísicas.
Resultados prévios indicam que o modelo baseado em redes neurais convolucionais é capaz
de agregar informações de alta frequência às inversões sísmicas.

\subsection{Objetivos Específicos}
Os objetivos específicos deste trabalho são:
\begin{itemize}
 \item Conceber um modelo de rede neural convolucional para super-resolução de imagens de inversão.
 \item Entender as abstrações em cada camada da rede neural.
 \item Estudar o espectro de frequência das imagens pós-processadas com o modelo de rede convolucional.
 \item Definir uma estratégia de aplicação do modelo para diferentes tipos de dados de inversão (parametrização).
 \item Estudo de incerteza do processo de super-resolução.
\end{itemize}

Um objetivo secundário deste trabalho é utilizar o conhecimento acumulado para estudar estratégias que
relacionem as redes convolucionais e a simulação geoestatística multiponto. Esta etapa de trabalho será desenvolvida
mediante aprovação do processo de doutorado sanduíche, em tramitação no CNPq com número 202482/2017-0.
Em caso de aprovação, esta atividade será realizada em cooperação com o Departamento
de Ciências Geológicas, Universidade Stanford, sob orientação do Prof. Dr. Jef Karel Caers.

\section{Organização do Texto}

Este documento está organizado da seguinte forma. Após esta breve introdução, o
Capítulo \ref{cap:2fundamentacao} apresenta a fundamentação teórica para os
processos de inversão sísmica, as redes neurais 
convolucionais e a super-resolução. O Capítulo \ref{cap:3revisaoliteraria} 
apresenta o estado da arte relacionado à inversão acústica  e
redes convolucionais aplicadas na geração de imagens em super-resolução. 
O Capítulo 4 %\ref{cap:3modeloHibrido} 
trata dos resultados preliminares, complementação da proposta e o plano de trabalho.
% %Após o retorno ao Brasil, estão planejados mais 8 meses 
% %de trabalho para finalizar a escrita da tese e defesa.
