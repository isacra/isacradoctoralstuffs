\chapter{Introdução}
\label{cap:1intro}
\pagenumbering{arabic}

Um aspecto importante nas ciências físicas é poder inferir sobre parâmetros
físicos a partir de dados. Em geral, as leis da física disponibilizam os
artefatos necessários para calcular valores de dados, a partir de um modelo.
Este procedimento é conhecido como problema direto (\textit{forward problem}).
A modelagem direta, portanto, inicia com um modelo, sobre o qual um experimento ou processo
é simulado matematicamente. Se o modelo estiver correto, a resposta
obtida deve parecer com dados reais. O processo de inversão faz exatamente o contrário,
consiste em utilizar as medidas efetuadas para inferir os valores de parâmetros que
caracterizam o sistema \citep{tarantola}.

Para entender o problema inverso é conveniente explicar o problema direto antes.
Considere o seguinte exemplo: suponha que uma pedra é lançada em um poço de água.
Após determinado tempo um som é ouvido. É esperado que haja uma relação entre a
profundidade do poço e o tempo entre soltar a pedra e o som do impacto. Da física,
ocorre a existência de uma relação causa-efeito para este evento dada por:
\begin{equation}
\label{eq:forw}
T = \sqrt{\frac{1}{5}H} + \frac{1}{340}H
\end{equation}
onde a profundidade $H$ é a causa e tempo $T$ é o efeito.
Neste caso, o problema direto é calcular o tempo $T$ para uma dada profundidade $H$.
A solução pode ser determinada inserindo um valor pra $H$ em \ref{eq:forw} e calcular o
valor de $T$. O problema inverso é uma abordagem mais difícil, pois se deseja
saber a profundidade $H$, dado apenas o tempo $T$.

No exemplo citado, tanto o problema direto quanto o inverso possuem solução. Entretanto,
a maioria dos problemas inversos recai sobre duas características
comuns. Primeiro, a não unicidade de solução (problema não-determinístico), ou seja, o mesmo conjunto de medidas
observáveis pode resultar de mais de uma configuração de parâmetros. Segundo,
a natureza mal-posta do problema inverso, isto é, uma pequena mudança arbitrária nos
valores observados pode causar uma mudança grande da solução fonte equivalente.

Por conta da sua característica mal-posta, o problema inverso
possui muitas soluções possíveis, de modo que representaria um equívoco considerar apenas
uma solução como a mais correta. Via de regra, ao final do processo de inversão é comum
realizar um processo de amostragem dentro do conjunto das possíveis soluções a fim
de obter um estudo sobre elas. Este estudo pode ser, por exemplo, uma análise de 
incerteza em todo da média das soluções do problema inverso.

O problema inverso possui um papel extremamente importante em diferentes áreas do conhecimento,
como matemática, medicina, física e Geoestatística. Geoestatística é a aplicação de métodos
estatísticos nas ciências da terra. Esta é a ciência que trata, por exemplo, da modelagem
e caracterização de reservatórios, cujo tema é de amplo interesse para a indústria de óleo
e gás. Por caracterização de reservatório se entende o processo para obter um modelo de propriedades
petrofísicas (por exemplo, tipos de contato entre rochas, porosidade e permeabilidade),
em 3-D e alta resolução, que seja consistente com os dados de que se dispõe.
O processo de caracterização de reservatórios possui diferentes
etapas. Estas etapas podem ser descritas em alto nível como na
abordagem sequencial a seguir:
\begin{enumerate}
 \item A primeira etapa envolve da geometria e estratigrafia dos intervalos do reservatório
 a ser modelado. Ainda, o desenvolvimento de um modelo conceitual de continuidade para
 propriedades como fácies, porosidade e permeabilidade \label{itm:1}.
 \item Os tipos de contato (interfaces / \textit{facies}) entre rochas são modelados\label{itm:2}.
 \item A propriedade porosidade é modelada com base nos tipos de facies, antes da permeabilidade.
 Isto é possível devido à disponibilidade de dados sísmicos e dados amostrais localizados, também chamados de poços\label{itm:3}.
 \item Os modelos 3-D para permeabilidade são atrelados à porosidade e facies anteriormente estabelecidos\label{itm:4}.
 \item Múltiplas realizações igualmente prováveis são realizadas por repetição de todo o processo. Embora todas as
 realizações sejam equiprováveis, há realizações mais similares a outras, de modo que a classe à qual pertencem possui
 maior probabilidade\label{itm:5}.
 \item Os modelos são usados como entrada em um simulador ou visualizados e usados como suporte na tomada de decisão\label{itm:6}.
\end{enumerate}

Embora a abordagem anterior utilize alguns termos que, à primeira vista, pareçam incompreensíveis, sua apresentação 
serve para contextualizar que o processo de inversão no domínio da modelagem de reservatório,
é realizado na etapa \ref{itm:3}. Porque os dados medidos são obtidos por sísmica de reflexão,
este modelo de inversão é chamado de inversão sísmica. A inversão sísmica na
modelagem de reservatórios disponibiliza artefatos que são modelos de propriedades de rocha
(propriedades petrofísicas) a partir, principalmente, mas não exclusivamente,
da sísmica disponível e de modelos construídos com dados amostrais. Tais artefatos
são visualizados na forma de imagens, de modo que é possível supor que quanto maior
o nível de resolução destas imagens, mais contundente será a justificativa para a tomada de decisão (\ref{itm:6}).

Este trabalho aborda busca exatamente atacar a problemática da super-resolução dos artefatos da inversão sísmica,
por meio de soluções baseadas na técnica de \textit{Redes Neurais Convolucionais}.
A abordagem se dá por meio da incorporação deste modelo de rede neural á última
etapa do processo de inversão sísmica, as realizações, para alcaçar um maior nível de
resolução das imagens de proriedades petrofísicas obtidas por meio de inversão sísmica.

\section{Objetivo}

O objetivo do presente trabalho consiste em propor um modelo fluxo de trabalho
para integração de um modelo para obter imagens de inversão sísmica em hiper-resolução.
Resultados prévios indicam que o modelo baseado em redes neurais convolucionais é capaz
de agregar informações de alta frequência às inversões acústicas.

Outro objetivo deste trabalho é desenvolver um modelo baseado em redes neurais convolucionais que permita a
realização de simulação geoestatística multiponto. Esta etapa de trabalho será desenvolvida
em cooperação com o Departamento de Ciências Geológicas, Universidade Stanford, sob
orientação do Prof. Dor. Jef Karel Caers.


\section{Organização do Texto}

Este documento está organizado da seguinte forma. Após esta breve introdução, o
Capítulo %\ref{cap:2modelosSuperresolucao}
apresenta o estado da arte em modelos de
inversão sísmica. O Capítulo %\ref{cap:2modelosSuperresolucao}
apresenta o estado da arte 
em geração de imagens em super-resolução a partir de imagens de baixa resolução.
O Capítulo trata dos métodos de simulação geoestatística multiponto. O
Capítulo %\ref{cap:3modeloHibrido} 
trata da proposta do projeto e resultados
preliminares referente ao modelo de super-resolução treinado e aplicado às imagens
de impedância pós-inversão. Após o retorno ao Brasil, estão planejados mais 8 meses
de trabalho para finalizar a escrita da tese e defesa.
