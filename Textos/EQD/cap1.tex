\chapter{Introdução}
\label{cap:1intro}
\pagenumbering{arabic}

Um aspecto importante nas ciências físicas é poder inferir sobre parâmetros
físicos a partir de dados. Em geral, as leis da física disponibilizam os
artefatos necessários para calcular valores de dados, a partir de um modelo.
Este procedimento é conhecido como problema direto (\textit{forward problem}).
A modelagem direta, portanto, inicia com um modelo, sobre o qual um experimento ou processo
é simulado matematicamente. Se o modelo estiver correto, a resposta
obtida deve parecer com dados reais. O processo de inversão faz exatamente o contrário,
consiste em utilizar as medidas efetuadas para inferir os valores de parâmetros que
caracterizam o sistema \citep{tarantola} e muitas vezes se caracteriza
por ser não determinístico.

Para entender o problema inverso é conveniente explicar o problema direto antes.
Considere o seguinte exemplo: suponha que uma pedra é lançada em um poço de água.
Após determinado tempo um som é ouvido. É esperado que haja uma relação entre a
profundidade do poço e o tempo entre soltar a pedra e o som do impacto. Da física,
ocorre a existência de uma relação causa-efeito para este evento dada por:
\begin{equation}
\label{eq:forw}
T = \sqrt{\frac{1}{5}H} + \frac{1}{340}H
\end{equation}
Na qual $H$ é a causa e $T$ é o efeito.
Neste caso, o problema direto é calcular o tempo $T$ para uma dada profundidade $H$.
A solução pode ser determinada inserindo um valor pra $H$ em \ref{eq:forw} e calcular o
valor de $T$. O problema inverso é uma abordagem mais difícil, pois se deseja
saber a profundidade $H$, dado apenas o tempo $T$.

No exemplo citado, tanto o problema direto quanto o inverso possuem solução. Entretanto,
a maioria dos problemas inversos recai sobre duas características
comuns. Primeiro, a não unicidade de solução (não-determinístico), ou seja, o mesmo conjunto de medidas
observáveis pode resultar de mais de uma configuração do coração doente. Segundo,
a natureza mal-posta do problema inverso, isto é, uma pequena mudança arbitrária nos
valores observados pode causar uma mudança grande da solução fonte equivalente

O problema inverso possui um papel extremamente importante em diferentes áreas do conhecimento.
Com ele é possível por exemplo....


Por conta da sua característica mal-posta, já mencionada anteriormente, o problema inverso
possui muitas soluções possíveis, de modo que representaria um equívoco considerar apenas
uma solução como a mais correta. Via de regra, ao final do processo de inversão é
se realiza um processo de amostragem dentro do conjunto das possíveis soluções a fim
de obter um estudo sobre elas. Este estudo pode ser, por exemplo, uma análise de 
incerteza em todo da média das soluções do problema inverso.

No campo da geoestatística o processo de inversão tem o papel de caracterização
de reservatórios. Estes reservatórios podem ser....

O processo de caracterização de reservatórios possui diferentes
etapas. A etapa de inversão visa....
O processo inverso disponibilizam artefatos que...


Assim, este documento propõe a incorporação de um método de um modelo
baseado em aprendizagem de máquina ao processo de inversão sísmica.
A incorporação deste processo objetiva um ganho de resolução ....


\section{Objetivo}

O objetivo do presente trabalho consiste em propor um modelo fluxo de trabalho
para integração de um modelo para obter imagens de inversão sísmica em hiper-resolução.
Resultados prévios indicam que o modelo baseado em redes neurais convolucionais é capaz
de agregar informações de alta frequência às inversões acústicas.

Outro objetivo deste trabalho é desenvolver um modelo baseado em redes neurais convolucionais que permita a
realização de simulação geoestatística multiponto. Esta etapa de trabalho será desenvolvida
em cooperação com o Departamento de Ciências Geológicas, Universidade Stanford, sob
orientação do Prof. Dor. Jef Karel Caers.


\section{Organização do Texto}

Este documento está organizado da seguinte forma. Após esta breve introdução, o
Capítulo \ref{cap:2modelosSuperresolucao} apresenta o estado da arte em modelos de
inversão sísmica. O Capítulo \ref{cap:2modelosSuperresolucao} apresenta o estado da arte 
em geração de imagens em super-resolução a partir de imagens de baixa resolução.
O Capítulo trata dos métodos de simulação geoestatística multiponto. O
Capítulo \ref{cap:3modeloHibrido} trata da proposta do projeto e resultados
preliminares referente ao modelo de super-resolução treinado e aplicado às imagens
de impedância pós-inversão. Após o retorno ao Brasil, estão planejados mais 8 meses
de trabalho para finalizar a escrita da tese e defesa.

