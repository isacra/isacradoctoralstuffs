\documentclass[a4paper,10pt]{report}
\usepackage[utf8]{inputenc}

% Title Page
\title{A convolutional model to predict forward velues of the sin function}
\author{Isaac Sacramento}


\begin{document}
\maketitle

\begin{abstract}
\end{abstract}

\chapter{Introduction}

Convolutional Neural Networks (CNNs) are specialized kind of neural network in the
of deep learning methods for processing data that has a known,
grid-like topology. They have been widely used to solve problems
which objective is to classify elements in a certain domain.
The name \textit{convolutional neural network} is related to the
fact that these networks employ a linear operation called convolution,
instead of ordinary matrix multiplication.
Different convolutional models are build each day to solve specific
classification problems, such as pointing a handwritten digit and classifying objects
in static and dynamic scenes.
On the other hand, there are certain domains in which a set of
images represent a spacial distributed phenomenon, such as geo-statistic
phenomenon. In this kind of event, points in an image area spacial correlated,
in such a way that, in order to stud it, one must to take into consideration
vertical and/or horizontal correlation between the points.

Using shallow neural networks on the study of spacially distributed phenomenon
may lead to not accurately measurement of the spacial correlation between the
points, once the process of feature extraction is based on ordinary
matrix multiplication.

\chapter{Regressive Convolutional Neural Networks}

\section{Predição da Função Seno}
As an introductory experiment to model a regressive CNN, the sin function
is estimated based on the convolution of 1D filters with n previews steps
of the same function. This function was arbitrarily chosen to represent the
a non-linear training set, but simples enough to make it feasible obtaining
a regressive convolutional model, instead of a classification one.
It is reasonable that this model must be improved and adapted to run over
real world datasets

\subsection{Conjunto de Dados}
On supervised training, to predict the sin function through a shallow
neural network, the input dataset is composed by the of angles  $x$
in the interval [$-k\pi,k\pi$], $k >= 0$. Considering the input
dataset for convolutional neural networks are training images, one
expects to be possible recognizing spacial patterns from which the
values of the sin function can be obtained. In this context the training
image dataset was created as a matrix in wich each row represents a 1D
training image composed by $10$ subsequent values of the function, the
network is then trained to predict the $11$th value. A Figura \ref{}
ilustrates the way how the experimental dataset was defined.

É importante salientar que, neste caso os valores dos ângulos de entrada para a função
seno não são utilizados como entrada da rede convolucional, de modo que a predição
deverá ocorrer baseada no entendimento do padrão da curva existente no conjunto de
de $10$ valores da função. Todo o conjunto é inicialmente por $90$ imagens.

\end{document}          
