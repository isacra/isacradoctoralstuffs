\chapter{Fundamentação Teórica}
\label{cap:2fundamentacao}
Neste capítulo serão apresentados os conceitos
abordados neste trabalho. O problema inverso será apresentado
em linhas gerais e o leitor tomará conhecimento do processo de inversão sísmica.
Serão apresentados os conceitos relacionados a \textit{Deep Learning}, assim como os
elementos de redes neurais convolucionais. Esta fundamentação
teórica é relevante para o entendimento de como o modelo de rede neural convolucional
pode ser adotado para obter ganho qualitativo e quantitativo no pós-processamento
da inversão sísmica.

\section{Problema Inverso}

A teoria de inversão é utilizada em diversas áreas para inferir os valores de
parâmetros relacionados com processos físicos a partir de um conjunto de dados medidos,
também chamados de dados experimentais. É possível descrever o problema inverso
como o processo de obter informações de um sistema parametrizado, a partir de
dados observáveis e das relações teóricas com os parâmetros não
observáveis e, quando possível, algum conhecimento \textit{a priori}.

Um sistema físico depende do domínio em estudo. Pode ser uma galáxia para um
astro-físico, pode ser a Terra para um geofísico ou uma partícula quântica
para um físico quântico. Em comum, o fato de que, para ser estudado, um sistema
físico segue três passos básicos: a parametrização do sistema, a modelagem direta e a modelagem inversa \cite{tarantola}.
A parametrização do sistema se refere à definição do conjunto mínimo de elementos (parãmetros, variáveis)
cujos valores caracterizam completamente o sistema. Como mencionado no Capítulo de introdução,
a modelagem direta se refere a definição das leis físicas que permitem realizar previsão
de dados observáveis, a partir de valores dos parâmetros do modelo. A modelagem inversa,
por sua vez, se caracteriza pelo uso de resultados atuais das medições dos parâmetros
físicos observáveis, para inferir os valores atuais dos parâmetros do modelo.

Resolver o problema direto significa prever os valores dos parâmetros observáveis (dados $d$),
que correspondem a um dado modelo (conjunto de parâmetros $m$). Esta predição pode ser denotada
pela Eq. \ref{eq:frdmdl}. Onde $F(.)$ é chamado operador direto.
\begin{equation}
\label{eq:frdmdl}
F(m) = d 
\end{equation}

O problema inverso pode ser descrito em uma forma discreta como:
\begin{equation}
\label{eq:deqgm}
m = F^{-1}(d)
\end{equation}
onde, F é o sistema físico investigado, e relaciona os parâmetros do modelo $m=(m_1, m_2,...,m_n) \subset R^n$
estimado com os dados observados $d \in R^s$.
Como mencionado, um problema inverso possui múltiplas soluções,
deste modo o modelo $m$ pertence a um conjunto de modelos $M$ admissíveis.
Na prática, $d$ pode ser uma função no domínio do tempo e/ou espaço, ou pode ser
uma coleção de observações discretas.

\section{Inversão Sísmica}

Os métodos geofísicos frequentemente envolvem a solução e avaliação de problemas inversos,
pois permitem inferir a distribuição das propriedades físicas na subsuperfície da Terra
usando observações da superfície. A inversão sísmica tem um papel fundamental na solução 
de problemas geofísicos, em especial na caracterização de reservatórios \cite{Bosch2010} \cite{Srivastava2009}.
Do ponto de vista prático, as soluções para o problema de inversão sísmica melhora a exploração e
o gerenciamento na indústria petrolífera, pois os dados sísmicos estimados possuem forte correlação com as
propriedades petrofísicas (porosidade, densidade, etc.) das rochas da subsuperfície \cite{Figueiredo2014}.

O método de aquisição sísmica de reflexão utiliza
pulsos sísmicos de uma fonte artificial controlada e monitora a resposta em
função do tempo. Neste sistema, cada região de contato entre dois tipos de rochas
diferentes gera reflexão e refração do pulso sísmico, como demonstrado na Figura
\ref{fig:1sismica}.
De um ponto de vista bastante elementar, é possível imaginar que a parte refletida da onda se
propaga em todas as direções, de modo que os componentes horizontal e vertical podem ser medidos.
O componente horizontal (\textit{s-wave}), referente à reflexão horizontal
da onda, é utilizada no processo de inversão conhecido como inversão elástica. Por outro lado, o componente
vertical da onda (\textit{p-wave}), referente à reflexão vertical do pulso emitido, é utilizado no processo
conhecido como inversão acústica.

\begin{figure}[ht!]
\begin{center}
  \includegraphics[width=0.8\textwidth]{fig/seismic_survey_2}
  \caption{Método de sísmica de reflexão \citep{figsismica}}
  \label{fig:1sismica}
\end{center}
\end{figure}

O pulso de onda emitido durante a aquisição sísmica possui um formato próprio, uma identidade, 
conhecido como \textit{wavelet}. É possível imaginar que a resposta sísmica medida
é composta em parte por esta identidade e, em parte, pela característica da região de contato
entre duas camadas de rochas diferentes, na qual o pulso reflete.
Esta característica é chamada de coeficiente de refletividade (equação \ref{eq:refletv}):
\begin{equation}
r(t) = \frac{z(t+\delta t)-z(t)}{z(t+\delta t)+z(t)}
\label{eq:refletv}
\end{equation}
onde, $z(t)$ é a impedância acústica no tempo $t$ definida por
$z(t)=\rho(t)v(t)$, onde $\rho(t)$ é a densidade da rocha e $v(t)$ a
velocidade de propagação da onda acústica.
O dado sísmico utilizado na inversão acústica, portanto,
é uma aproximação da resposta da camada terrestre;
a convolução entre a wavelet de aquisição e o valor de refletividade entre as
camadas da subsuperfície, com ângulo de incidência e reflexão de $90^\circ$,
respectivamente. Por este motivo, este modelo é chamado convolucional.
Com os coeficientes de reflexão e a discretização da medida de tempo, é possível
modelar o dado sísmico $d(t)$ aplicando a convolução $\otimes$
da \textit{wavelet} $s$ com os coeficientes de refletividade $r$:
\begin{equation}
d(t) = s(\tau) \otimes \sum{j-1}{N} r(t- t_j) \delta(t - t_j) + e_d(t)
\end{equation}
onde $N$ é o número total de camadas, $e_d(t)$ representa o ruído aleatório em função do tempo
e cada $d_{xy}$ é chamado de traço sísmico. Um conjunto de traços
sísmicos também é chamado de uma imagem, seção ou cubo, no caso de um
levantamento 3D. A \textit{wavelet} ideal seria um pulso tipo delta contendo
todas as frequências, entretanto, na prática as
\textit{wavelets} são pulsos de banda limitada entre $6Hz$ e $65Hz$, o que
limita a frequência da sísmica e sua resolução \citep[p. 11]{sen_livro}.
Como consequência, as imagens resultantes do processo de inversão também terão
o seu espectro de frequência limitado.
A Figura \ref{fig:wavelet} ilustra uma \textit{wavelet} típica extraída de dados
reais.

\begin{figure}[htp]
\begin{center}
  \includegraphics[width=0.8\textwidth]{fig/wavelet}
  \caption{\textit{Wavelet} extraída de dados reais}
  \label{fig:wavelet}
\end{center}
\end{figure}

\section{Redes Neurais Convolucionais}
Nesta seção serão apresentados os principais conceitos relacionados às redes
neurais convolucionais, sua estrutura e as principais
aplicações deste modelo de aprendizagem de máquina.

As Redes Neurais Convolucionais (CNN), também chamadas de redes convolucionais,
são um tipo de rede neural especializada em processamento de dados que possuem uma
topologia conhecida e em forma de grade \cite{Gdfl16}. Exemplos deste tipo de dado são as séries
temporais, que podem ser vistas como uma grade em uma dimensão (1D) com amostras
em intervalos de tempo regulares, e dados de imagem, que podem ser pensados como
uma grade 2D de \textit{pixels}. Este modelo de rede neural é chamada convolucional,
pois emprega a operação de convolução no lugar de multiplicação comum entre matrizes,
em pelo menos uma de suas camadas.

\subsection{Convolução}
A operação de convolução é definida como a integral do produto de duas funções após uma delas sofrer um
certo deslocamento. Considere um exemplo em que se deseja rastrear a localização de uma
nave espacial com um sensor a lazer. O sensor disponibiliza uma saída $x(t)$ referente à posição da nave
no tempo $t$. Ambos, $x$ e $t$, são valores reais, de modo que uma saída diferente pode ser obtida
em qualquer instante de tempo. Considerando que o sensor possui um certo ruido, para realizar uma
estimativa mais precisa da posição da nave é possível ponderar várias medidas de posição juntas.
Como os valores medidos mais recentemente são mais relevantes, é possível estimar uma função peso
$w(a)$, onde $a$ é o tempo de medição. Se esta média ponderada for aplicada a todos os instantes,
a estimativa de posição da nave será suavizada:

\begin{equation}
 s(t) = \int{x(a) w(t-a)da}
 \label{eq:1}
\end{equation}

Esta operação é chamada convolução e pode ser definida para quaisquer
funções para as quais a integral da equação \ref{eq:1} esteja definida. A convolução
costuma ser denotada com um asterisco e aplicada com
o tempo discretizado, de modo que o tempo $t$ é assumido como valores inteiros:
\begin{equation}
 s(t) = (x * w)(t) = \sum{a=-\inf}{\inf}{x(a)w(t-a)}
 \label{eq:2}
\end{equation}

No contexto das redes convolucionais, $x$ se refere ao conjunto de imagens de entrada, uma sequência multidimensional
de dados, e $w$ é denominado \textit{kernel} ou filtros, uma sequência multidimensional de parâmetros 
a serem otimizados pelo algoritmo de aprendizagem.
Nos casos em que o problema compreende imagens $I$ e filtros $K$ utilizados em duas dimensões
a convolução ganha o seguinte formato:

\begin{equation}
 S(i,j) = (I*K)(i,j) = \sum{m}\sum{n}{I(m,n)K(i-m,j-n)}
\end{equation}

Nas CNN há pelo menos duas estruturas básicas, a camada convolucional e a camada de \textit{pooling}.
A arquitetura típica de uma CNN compreende duas camadas convolucionais, cada uma das quais é seguida de
uma camada \textit{pooling} como ilustrado na figura \ref{fig:cnn_basic_arq}. As imagens 
de entrada se tornam menores à medida que progridem ao longo da rede,
mas se tornam mais profundas em termos de hierarquia de conceitos extraídos. No topo da pilha de camadas
são adicionados um conjunto de camadas completamente conectadas e a última camada onde ocorre
a saída prevista. Esta estrutura de camadas completamente
conectadas é a mesma utilizada nas redes neurais tradicionais do tipo \textit{feedforward}, nas quais todos
os nerônios de uma camada estão conectados a todos os neurônios da camada seguinte. 
\begin{figure}[htp]
\begin{center}
  \includegraphics[width=0.8\textwidth]{fig/cnn_basic_arq}
  \caption{Arquitetura típica de uma rede neural convolucional. Fonte:\cite{aurelien17}}
  \label{fig:cnn_basic_arq}
\end{center}
\end{figure}

A camada convolucional é o elemento mais importante de uma CNN. Esta camada é estruturada
de modo a fazer com que cada um dos seus neurônios esteja conectado a um 
pequeno grupo de \textit{pixels} da camada de entrada (figura \ref{fig:cnn_arq}) e não a todos os \text{pixels}, como
ocorre em redes neurais tradicionais. Cada neurônio da camada seguinte se conecta apenas a neurônios
contidos em uma pequena região da camada anterior e assim sucessivamente, esta região que define
o grupo de neurônios conectados ao neurônio da próxima camada é chamada \textbf{campo perceptivo}.
Este formato permite o aprendizado de características de baixo nível na primeira camada e de
características de mais alto nível nas camadas seguintes.
\begin{figure}[htp]
\begin{center}
  \includegraphics[width=0.8\textwidth]{fig/cnn_arq}
  \caption{Camadas de uma CNN com campos receptivos retangulares.}
  \label{fig:cnn_arq}
\end{center}
\end{figure}

A figura \ref{fig:cnn_stride} ilustra a conexão entre as camadas de uma rede convolucional.
Considere um neurônio localizado na linha $i$ e coluna $j$ de uma dada camada.
Este neurônio estará conectado às saídas dos neurônios da camada anterior
localizados nas linhas $\times{i}{s_h}$ até $\times{i}{s_h}+f_h - 1$, colunas
$\times{j}{s_w}$ até $\times{j}{s_w}+f_w - 1$, onde
$f_h$ e $f_w$ são a altura e a largura do campo receptivo, $s_h$ e $s_w$
são os deslocamentos vertical e horizontal ao longo das imagens da camada anterior.
O tamanho destes deslocamentos é chamado de passo ou \textit{stride}
e quanto maior o \textit{stride}, menor será a imagem resultante na camada seguinte. \textit{Stride} de tamanho
$0$ faz com que a camada seguinte tenha as mesmas dimensões da camada anterior.
\begin{figure}[htp]
\begin{center}
  \includegraphics[width=0.8\textwidth]{fig/cnn_layer_stride_2}
  \caption{Conexão entre camadas com campo receptivo 3 x 3 e \textit{strides} de tamanho 2.}
  \label{fig:cnn_stride}
\end{center}
\end{figure}

\subsection{Filtros}
Os pesos dos neurônios em uma camada convolucional podem ser representados como uma pequena
imagem do tamanho do campo receptivo. Estes filtros (pesos) são os elementos
convolvidas com a imagem de entrada para obter o resultado da camada convolucional.
A figura \ref{fig:conv_filt} ilustra dois conjuntos de pesos possíveis. O primeiro filtro é um quadrado preto
(\textit{pixel} de valor 0) contendo uma coluna central branca (\textit{pixels} com valor 1). 
Analogamente, o segundo filtro é um quadrado preto contendo uma linha central branca.
É possível notar na imagem da esquerda que as linhas verticais brancas se tornaram mais
evidentes enquanto o restante se tornou mais borrado. De modo análogo, na imagem da direita,
a convolução com o filtro horizontal evidenciou as linhas brancas horizontais, ao passo que
o restante ficou borrado. Assim, ao convolver uma entrada com o mesmo conjunto de filtros
da camada convolucional, se obtém o mapa de características (feature map).
\begin{figure}[htp]
\begin{center}
  \includegraphics[width=0.7\textwidth]{fig/conv_filt}
  \caption{Aplicação de dois filtros diferentes para obter mapas de características.}
  \label{fig:conv_filt}
\end{center}
\end{figure}

O exemplo anterior apresenta a convolução de uma imagem com dois filtros possíveis, em uma representação 2D.
Entretanto, em situações reais a camada convolucional possui muitos mapas de características, resultando
em uma representação em 3D como ilustrado na figura \ref{fig:featmaps}. O mapa de características
de uma camada convolucional é o resultado da convolução de uma das imagens de entrada com os diversos
filtros específicos desta camada, os quais são iniciados, na maior parte dos casos, aleatoriamente.
Na figura estão ilustrados os mapas para a convolução com apenas uma imagem, de modo que é possível
imaginar que à medida que o número de imagens aumenta, a estrutura ilustrada se replica horizontalmente. 
\begin{figure}[htp]
\begin{center}
  \includegraphics[width=0.7\textwidth]{fig/feat_maps}
  \caption{Camadas convolucionais com múltiplos mapas de características e imagens com três canais.}
  \label{fig:featmaps}
\end{center}
\end{figure}

As redes convolucionais se sustentam sobre três pilares: interações esparsas, compartilhamentos
de parâmetros e representações equivalentes. 
As interações esparsas, também chamadas de conectividade esparsa ou pesos esparsos,
ocorre quando os filtros possuem  dimensão menor que a entrada, ou seja a
dimensão do campo receptivo é menor que a dimensão das imagens de entrada.
De um ponto de vista prático, a imagem de entrada pode ter milhares de \textit{pixels}, entretanto, é 
possível detectar apenas pequenas regiões com características de maior relevância
com filtros que compreendam apenas algumas dezenas ou centenas de \textit{pixels} na imagem.
Por exemplo, é possível identificar características de uma face humana na identificação de pessoas, ou estruturas com
significado geológico em um estudo geofísico. Como consequência,
menos parâmetros são armazenados e há um ganho na eficiência estatística do
modelo. As figuras \ref{fig:sparse} e \ref{fig:full} ilustram
os modelos de conectividade esparsa e tradicional, respectivamente.
É possível notar que na conectividade tradicional (figura \ref{fig:full}) todos os elementos da camada inferior
afetam o elemento em destaque $s_3$ da camada seguinte, enquanto na conectividade esparsa (figura \ref{fig:sparse}) apenas
três elementos afetam o elemento em destaque. O número de elementos que afetam o elemento em destaque na
conectividade esparsa é definido pelo tamanho do filtro utilizado na convolução.

\begin{figure}[htp]
\begin{subfigure}{.5\textwidth}
  \centering
  \includegraphics[width=.9\linewidth]{fig/sparse}
  \caption{Conectividade esparsa.}
  \label{fig:sparse}
\end{subfigure}
\begin{subfigure}{.5\textwidth}
  \centering
  \includegraphics[width=.9\linewidth]{fig/full}
  \caption{Conectividade tradicional.}
  \label{fig:full}
\end{subfigure}%
\end{figure}

O \textbf{compartilhamento de parâmetros}, também chamado de \textbf{pesos amarrados} 
em uma rede convolucional, se refere ao uso do mesmo parâmetro para mais de uma função no modelo.
Nas redes neurais tradicionais, cada elemento da matriz de pesos é usado apenas uma vez quando a
saída da camada é calculada, pois é multiplicado por apenas um elemento da entrada. No compartilhamento
de pesos, o valor do peso aplicado a uma entrada está relacionado ao valor de um peso aplicado em
algum outro local. Assim, cada elemento do filtro é usado em toda posição da entrada,
de modo que, ao invés de aprender um conjunto separado de parâmetros para toda localização da imagem, apenas
um conjunto é aprendido. Isto faz com que a convolução seja mais eficiente que a multiplicação de matriz
do ponto de vista de requisitos de memória e eficiência estatística.

\subsection{Pooling}
Uma camada em uma rede convolucional consiste de três estágios. No primeiro estágio,
a camada realiza diversas convoluções para produzir um conjunto de ativações lineares.
O segundo estágio é chamado etapa de detecção, na qual cada ativação é submetida a uma
função não-linear. A terceira etapa é chamada de \textit{pooling}, responsável por
modificar a saída para o resumo estatístico das saídas em uma determinada vizinhança. A operação de
\textit{pooling} permite tornar invariante pequenas translações no conjunto de entrada,
ou seja, ainda que haja pequenas translações na entrada, os valores da maioria das saídas após a
o \textit{pooling} permanecem iguais. A figura \ref{fig:pool} ilustra o funcionamento da função de \textit{pooling}.
\begin{figure}[htp]
\begin{center}
  \includegraphics[width=0.7\textwidth]{fig/pool}
  \caption{Operação de \textit{pooling} com região de tamanho 3. Nesta operação é selecionado o máximo valor de ativação da etapa de detecção.}
  \label{fig:pool}
\end{center}
\end{figure}

A operação de \textit{pooling} permite lidar com entradas de tamanho variável.
Classificar imagens de tamanhos diferentes, por exemplo, pode ser realizado
variando o tamanho entre as regiões de \textit{pooling} de modo que a camada de 
de classificação sempre receba o mesmo número de sumários estatísticos
independente do tamanho da imagem.
