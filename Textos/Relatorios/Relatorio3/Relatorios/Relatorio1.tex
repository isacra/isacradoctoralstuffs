% $Id: AllegProposal.tex,v 1.8 2000/07/05 21:02:12 culver Exp $
% AllegProposal.tex
% by A. Thall
% 13. Feb 2003
%
% Small edits and a few additions made by R. Roos
% 21 Jan 2007
% Most particularly, the "box" around the thesis statement has been removed,
% section titles have been modified. The section named "Prior work II" has
% been commented out. The \topmargin has been changed to -.5in and the
% change to \parindent has been commented out.
% The filename "nausicaa.eps" has been changed to simply "nausicaa" so that
% pdflatex can be used on the file (and a file named "nausicaa.pdf" has
% been created using the "epstopdf" command).
% Several subsections have been added to illustrate subsection usage.
% The word "comp" has been replaced by "project" or "thesis" throughout.
% Other small changes have been made.
%
% This document provides a sample Senior Project Proposal template for use
% by students in Allegheny's CS and Applied Computing programs.

\NeedsTeXFormat{LaTeX2e}
\documentclass[11pt]{article}
\usepackage[utf8]{inputenc}
%The following is used by WinEdt to set up cross-referencing to the BibTeX files
%It is NOT commented out---the comment lets it be simply ignored by non-WinEdt LaTeX compilers

%GATHER{mybibtexDB.bib}

\usepackage{setspace}
\usepackage{amsmath}
\usepackage{amssymb}
\usepackage{epsfig}
\usepackage{fancybox}
\usepackage{listings}
\usepackage{algo}
\usepackage{url}

\setlength{\textheight}{9in}
\setlength{\textwidth}{6in}
\setlength{\oddsidemargin}{.25in}
\setlength{\topmargin}{-.5in}  % changed from -.25 by RSR on 1/21/07
%\parindent .5in    % commented out by RSR 1/21/07

%put words in the hyphenation statement if you want to enforce
%how LaTeX should break them (or not) at the end of a line.
%\hyphenation{repre-sen-tations problems exact linear}
\hyphenation{itself}

%%%%%
%% Commented out -- RSR, 1/21/07
%%%%%
% The following provides a box to surround the thesis statement
%\newenvironment{Thesis}%
%{\begin{Sbox}\begin{minipage}{.95\linewidth}}%
%{\end{minipage}\end{Sbox}\begin{center}\fbox{\TheSbox}\end{center}}

\title{Relatório Parcial: Ano 2}
\author{Isaac L.\ Santos\ Sacramento \\ Orientador: Mauro Roisenberg}

\begin{document}

% You can specify a language and other options for
% the code-formatting "listings" package
\lstset{language=C++,basicstyle=\small,
        stringstyle=\ttfamily,showstringspaces=false}

\singlespace
\maketitle

\begin{abstract}                % ~350 words max
Neste relatório são apresentadas as atividades realizadas do final do segundo ano até o início do terceiro ano de projeto, mais precisamente no
período de outubro de 2016 até abril de 2017. Neste período foram realizados estudos e experimentos relacionados à aplicação
de \textit{Redes Neurais Convolucionais} para realizar predição de propriedades geofísicas e hiper-resolução de imagens pós-inversão.
Estas atividades estão descritas nas seções a seguir.

\end{abstract}

\doublespace
% This sets section-numbering to only include Section and Subsection numbers
\setcounter{secnumdepth}{2}

\section{Introdução}

No período de outubro de 2016 até abril de 2017 foram exploradas duas novas frentes de trabalho relacionados ao uso
das redes neurais convolucionais no processo de simulação geoestatística e na hiper-resolução pós-inversão.
Após o estudo dos métodos de simulação multiponto, foram realizados experimentos com o intuito de obter um modelo de rede neural convolucional
capaz de aprender os padrões das imagens de treinamento e reproduzi-los posteriormente na geração de novas realizações estatísticamente compatíveis
com a realidade.

Os primeiros experimentos realizados neste período foram implementados 
com a biblioteca de aprendizagem de máquina (textit{Machine Learning}),
Tensor Flow. Esta biblioteca permite a implementação de modelos de redes
neurais convolucionais em linguagem Python.


\section{Revisão Bibliográfica}

Os métodos tradicionais de simulação se apoiam na modelagem de estatísticas em dois pontos, geralmente as covariâncias e variogramas. Muitos
fenômenos são complexos e inviabilizam a captura de seus padrões espaciais por meio de estatísticas de dois pontos. A simulação
geoestatística multiponto é um método genérico e se
baseia em três mudanças conceituais formalizadas por \cite{Guardiano1993}. A primeira, afirma que conjuntos de dados podem não ser suficiente
para inferir todas as características estatísticas que controlam o que se deseja modelar. A segunda é adotar uma estrutura estatística
não-paramétrica para representar a heterogeneidade. A terceira mudança conceitual é avaliar a estatística de eventos de dados de múltiplos
pontos. As estatísticas multipontos são expressas como funções densidades cumulativas para uma variável aleatória $Z(x)$ condicionadas
a eventos de dados locais $d_n =  {Z(x_1), Z(x_2),...,Z(x_n)}$, isto é, os valores de $Z$ nos nós vizinhos $x_i$ de $x$, equação \ref{eq:1}.
\begin{equation}
 f(z, x, d_n) = Prob({Z(x)<=z|x})
 \label{eq:1}
\end{equation}


\section{Redes Neurais Convolucionais na Predição da Função Seno}

O uso da função seno como experimento inicial buscou alcançar um modelo de
rede convolucional capaz de realizar a aproximação de função utilizando
imagens da função. A função seno foi arbitrariamente escolhida para
representar um conjunto de treinamento não-linear simples o
suficiente, que permitisse obter um modelo convolucional capaz de realizar predição
ao invés de classificação. É importante ressaltar que este modelo deve ser
aprimorado e adaptado para a resolução de problemas com conjuntos de dados
do mundo real.

Para prever a função seno com redes neurais convencionais, o
conjunto de entrada compreende valores $x$ no intervalo [$-k\pi,k\pi$],
$k >= 1$. Tendo em vista que o conjunto de entrada das redes neurais convolucionais
são imagens de treinamento a partir das quais se deseja extrais padrões espaciais
que permitam realizar predição, para este primeiro experimento foi gerada uma
matriz de senos. Nesta matriz de senos, cada linha representa uma imagem em 1D
composta por $10$ valores subsequentes da função seno, o valor seguinte é o que se
deseja prever. Desta forma, o experimento se assemelha à predição de uma série temporal.
É importante salientar, porém, que neste caso os valores dos ângulos de entrada para a função
seno não são utilizados como entrada da rede convolucional, de modo que a predição
deverá ocorrer baseada no entendimento do padrão da curva existente no conjunto de
de $10$ valores da função. Todo o conjunto é inicialmente composto por $90$ imagens.


\section{Super-resolução em Imagens Pós-inversão Sísmica}


\section{Conclusões}
A simulação multiponto gera realizações que reproduzam padrões estatísticos inferidos a partir de alguma fonte, usualmente uma imagem de treinamento. 
Como ferramenta de modelagem, os algoritmos de aprendizagem de máquina são universais, adaptativos, não-lineares, robustos e eficientes. 
Eles podem alcançar soluções aceitáveis para problemas de classificação, regressão e modelagem de densidade de probabilidade em espaço de alta dimensão 
e com características espacialmente referenciados. Dentre os raros trabalhos encontrados estão o uso de algoritmos genéticos na simulação de variáveis
categóricas para reproduzir estatísticas multipontos \cite{Peredo2012}. Este algoritmo 
requer alto desempenho computacional e, portanto, depende da disponibilidade de computadores com múltiplos núcleos, bem como unidades de processamentos 
gráficos, características inerentes aos métodos de algoritmos genéticos. A simulação multiponto com redes neurais apresentada por Jef Caers \cite{caers_1998} explora uma 
solução neural para a simulação pixel a pixel e, embora haja diversas citações a este trabalho, nenhuma delas está relacionada com a melhoria, expansão 
ou aplicação do método proposto. Este fato se dá, possivelmente, por conta do interesse no desenvolvimento de novos algoritmos.Com base no levantamento literário, se observa a possibilidade de explorar a simulação multiponto por meio da 
implementação baseada em aprendizagem de máquina. Os experimentos de simulação multiponto com os métodos de \textit{Deep Learning} estão em andamento e
parecem promissores para este fim, entretanto, ainda não apresentaram resultados conclusivos.

\nocite{*}

\begin{spacing}{1}
   \bibliographystyle{plain}
   \bibliography{mybibtexDB}
 \end{spacing}

\end{document}
\grid
